\section{Deployment}
\label{chap:deployment}

At this stage, we focus on how the developed models can actually be used in a real-world scenario. The goal is to smoothly integrate the best-performing model — Random Forest Regressor — into the ranking system of a rental aggregator platform. The idea is simple: use what we’ve built to help users see better offers first, especially when no reviews or previous data are available — the so-called cold start situation.

\paragraph{Deployment plan} –
The deployment is planned as follows:

Export the trained model into a serialized format (e.g., using \texttt{joblib} or \texttt{pickle}) for reuse in a production environment.

Create a lightweight API service (e.g., Flask or FastAPI) that receives a listing’s input data and returns a predicted rating.

Integrate this prediction service into the platform’s backend, so it can run in real-time or batch mode, depending on the traffic and system load.

Periodically retrain the model when more listings and real user feedback become available, to improve its accuracy over time.

This setup is simple enough to maintain, yet flexible if we decide to scale or swap out the model later.

\subsection{Limitations and Challenges }
\label{sec:limitation}

There were a few things we bumped into along the way:

Cold start still means guessing: Even with a good model, predicting without real user reviews will always have some uncertainty.

Limited compute resources: We couldn't try more advanced models (like neural networks) due to hardware constraints.

Sparse and messy data: Some fields (like amenities) were messy, and it took time to clean and transform them into usable features.

Geographical bias: Since we only worked with listings from Rio, the model might not generalize well to other locations without fine-tuning.


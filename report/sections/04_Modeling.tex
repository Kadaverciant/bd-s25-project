\section{Modeling}
\label{chap:modeling}
This stage of the project focuses on applying machine learning techniques to the structured dataset in order to develop a predictive system for estimating listing ratings. The aim is to select a model that not only performs well statistically, but is also interpretable and practical for deployment in a real-world recommendation pipeline, particularly under cold-start conditions where historical data is absent.

\subsection{Modeling techniques}
To explore the modeling space, we trained and evaluated two distinct approaches:

\textbf{Linear Regression } – A classic and widely used statistical model that assumes a linear relationship between the input features and the target variable. It serves as a strong baseline due to its simplicity, speed, and interpretability. Linear Regression attempts to fit a straight line that best describes the relationship between the predictors and the rating score..

\textbf{Random Forest Regressor } – A powerful ensemble method that aggregates multiple decision trees, capturing non-linear relationships and interactions between variables. It is well-suited for structured data and has strong generalization ability.

\paragraph{Preprocessing assumptions}
\begin{itemize}
    \item Categorical variables were transformed using One-Hot Encoding.
    \item The string \texttt{amenities} field was vectorized using a pre-trained \texttt{word2vec} model to preserve semantic relationships.
    \item Geolocation data was converted from latitude/longitude to ECEF coordinates to better encode spatial proximity.
\end{itemize}

These transformations ensured the models could effectively leverage all available information and handle heterogeneous feature types.

\subsection{Generate test design}
We split the dataset into an 80\% training set and a 20\% test set, ensuring a clean separation for unbiased evaluation. This split strategy allowed us to simulate real-world model behavior on unseen listings.

We evaluated models using two standard regression metrics:
\begin{itemize}
    \item \textbf{Root Mean Squared Error (RMSE)} – Emphasizes large errors, useful for detecting problematic predictions.
    \item \textbf{Mean Absolute Error (MAE)} – Measures average prediction error in the original rating scale, making it more interpretable.
\end{itemize}

\subsection{Build model }
Model training was optimized using cross-validation and grid search to explore hyperparameters and reduce overfitting. This ensured that the final models were not just fitted to training data but generalized well to new inputs.

\textbf{Model behavior:}  
Random Forest effectively captured non-linear relationships, especially between location, property characteristics, and host quality. Linear Regression, though limited in expressive power, served as a lightweight and interpretable baseline.

\subsection{Assess model }

Based on the results of the cross validation, we decided to choose the RF model as the main one. Below is a table of models and their results.

\begin{table}[ht!]
\small
\centering
\caption{Grid Search Results For Random Forest Regressor (RMSE)}
\label{tab:grid_search_rf_rmse}
\begin{tabular}{lll} 
\toprule
  & \textbf{Max Depth 5} & \textbf{Max Depth 10} \\
\midrule
\textbf{5 Trees} & 9.12 & 8.43 \\
\textbf{10 Trees} &  9.16 & \textbf{8.4} \\
\bottomrule
\end{tabular}
\end{table}

\begin{table}[ht!]
\small
\centering
\caption{Grid Search Results For Random Forest Regressor (MAE)}
\label{tab:grid_search_rf_mae}
\begin{tabular}{lll} 
\toprule
  & \textbf{Max Depth 5} & \textbf{Max Depth 10} \\
\midrule
\textbf{5 Trees} & 5.7 & 5.35 \\
\textbf{10 Trees} &  5.74 & 5.32 \\
\bottomrule
\end{tabular}
\end{table}

\begin{table}[ht!]
\small
\centering
\caption{Grid Search Results For Linear Regressor (RMSE)\\ Where $\lambda$-Elastic Net Parameter and $\beta$-Reg Parameter }
\label{tab:grid_search_lr_rmse}
\begin{tabular}{llll} 
\toprule
  & \textbf{$\lambda$= 0.0} & \textbf{$\lambda$=  0.5} & \textbf{$\lambda$=  1.0} \\
\midrule
\textbf{$\beta$= 0.01} & 9.13 & \textbf{9.0} & 9.12 \\
\textbf{$\beta$= 0.1} &  9.07 & 9.05 & 9.15 \\
\textbf{$\beta$= 1.0} &  9.02 & 9.28 & 9.32 \\
\bottomrule
\end{tabular}
\end{table}

\begin{table}[ht!]
\small
\centering
\caption{Grid Search Results For Linear Regressor (MAE)\\ Where $\lambda$-Elastic Net Parameter and $\beta$-Reg Parameter}
\label{tab:grid_search_lr_mae}
\begin{tabular}{llll} 
\toprule
  & \textbf{$\lambda$= 0.0} & \textbf{$\lambda$=  0.5} & \textbf{$\lambda$=  1.0} \\
\midrule
\textbf{$\beta$= 0.01} & 5.64 & 5.64 & 5.66 \\
\textbf{$\beta$= 0.1} &  5.62 & 5.69 & 5.7 \\
\textbf{$\beta$= 1.0} &  5.65 & 5.85 & 5.95 \\
\bottomrule
\end{tabular}
\end{table}

